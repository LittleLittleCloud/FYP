\documentclass[]{template}
\begin{document}
\section{结论与展望}
本文较为全面的考察了事件描述对事件结果的作用:第一章中,我们借助了多种分类手段比较了包含和不包含事件描述对事件结果预测的影响,并比较了在解释性较强的线性模型中,加入和去除事件描述对随机效应的确定系数的变化,从而证明了事件描述对事件举办结果起重要作用。第二章,我们使用了更复杂的模型来处理事件描述,通过比较三种主流的文本处理方式:线性回归,卷积神经网络和RNN组成的预测器,证明了使用RNN的方法在预测事件结果的问题上效果最好。第三章,我们使用生成对抗网络的结构,以变分自编码器为生成模型,带GRU的神经网络为判别模型的GAN\_PG来生成事件描述,实验结果证明GAN\_PG能够生成与真实事件描述接近的文本,并能够在原有事件描述的基础上改进事件描述。

但是本文提出的解决方法仍有诸多改进空间:首先,在事件结果的判断上有改进空间:本文对事件结果的划分依据为事件参与度,但更好的方式应该是来自参与者的评价或者参与者的活跃程度。其次,我们使用了事件结果作为事件描述的评价手段,即参与人数越高,事件描述越好。但是影响事件结果的并不只有事件描述,所以这种标注手段有较大的误差,最好的方式应该采用群智的手段,人工标注。最后,判别模型的编码器部分如果使用采样方式来获得隐编码,会具有更好的泛性,但本文仅使用了传统的RNN来编码。以上改进将作为下一步工作来完成。
\end{document}