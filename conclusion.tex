% \documentclass[]{template}
% \begin{document}
\xjsection{结论与展望}
\subsection{结论}
本文较为全面的考察了事件描述对事件结果的作用:第一章中,本文借助了多种分类手段比较了包含和不包含事件描述对事件结果预测的影响,并比较了在解释性较强的线性模型中,加入和去除事件描述对随机效应的确定系数的变化,从而证明了事件描述对事件举办结果起重要作用。第二章,本文使用了更复杂的模型来处理事件描述,通过比较三种主流的文本处理方式:线性回归,卷积神经网络和RNN组成的预测器,证明了使用RNN的方法在预测事件结果的问题上效果最好。第三章,本文使用生成对抗网络的结构,以变分自编码器为生成模型,带GRU的神经网络为判别模型的GAN\_PG来生成事件描述,实验结果证明GAN\_PG能够生成足够真实的事件描述。
\subsection{展望}
本文提出的解决方法仍有诸多改进空间:首先,可以通过使用来自参与者的评价或者参与者的活跃程度作为事件结果的划分标准,这样做将更符合人们对事件结果的判断的主观认知。其次,在标注事件描述的好坏时,可以使用群智的思路,利用人群对事件描述的评价来衡量事件描述的质量。最后,我们可以通过在判别模型的编码器部分加入采样环节,来获得隐编码。这样编码器将会具有更好的泛性。以上改进将作为下一步工作来完成。
% \end{document}