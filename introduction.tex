\documentclass[12pt]{template}
\usepackage{fancyhdr}

\begin{document}
\section{绪论} 
\par{
    在2014年前后,\textbf{Liu} 等学者提出了一个新的社交网络模型:\textit{EBSN}\cite{EBSN_linking},
在这篇论文中,\textbf{Liu.} 等把 \textit{EBSN} 分为线上和线下两部分,人们可以通过网络在线上互相交流,
但也可以通过参与线下活动的方式进行线下交流。这种新型的社交网络结构有很多有趣的性质,比如在事件的参与人数,组的
规模及其他层面上的长尾分布,以及较之于其他传统线上社交网络上的更为强烈的局部相关性。
}

\par{
    由于以上原因,越来越多的学者开始关注该领域,并大致分出了几个子方向:事件推荐,事件安排和其他。事件推荐更多
    的是从事件参与者的角度去考虑的,从事件参与者的兴趣标签出发,使用协同过滤,多标签聚类,随机游走或者其他的一些
    算法,来最大化某一个目标函数。由于网络规模等一些原因,获得全局最优一般是不现实的,因此相关的论文更多的是采用
    某些近似算法,来达到局部最优\cite{EBSN_event_reco}\cite{EBSN_on_social}\cite{EBSN_event_recom2}。
    事件安排的目的则是通过合理的安排每个事件的参与人员,在确保事件符合事件参与者的兴趣的同时,保证每个事件都符合
    组织者的预期:例如有足够的人参加。较之与事件推荐的不同,是事件安排同时要考虑两方面的因素,限制条件更多,例如
    活动参与人的限制,活动举办时间的限制和地点的限制,所以如何平衡好这方面的矛盾是解决这类问题的关键之一\cite{EBSN_conflict-aware_2016}
    \cite{EBSN_feedback-aware_2017}\cite{EBSN_conflict-aware_2015}。同样,想在这类问题上获得全局最优解也是很困难的,因此通过近
    似算法来得到局部最优解也是普遍的选择。
}
\par{
    除了以上两个方向,近年来关于EBSN的其他方向的研究也渐渐兴起。其中比较有趣的话题是通过研究\textit{EBSN}中人
    们如何选择和参与事件来帮助解释人群行为\cite{EBSN_understanding},以及研究话题模型在\textit{EBSN}中的传播
    方式。在\cite{EBSN_can_i}这篇论文中,由于普遍来说一个兴趣小组的三月存活率不到百分之30,因此作者通过研究组的创办者,组成员数量的增长速度以及其他因素,来预测最终该
    组是否能够存活。在\cite{EBSN_who_will}这篇论文中,作者把角度放在了预测事件的参与人员情况上。由于人们在选择
    事件时会有模式可寻,因此通过学习历史数据可以预测出哪些人会有可能对一个事件感兴趣。
}
\par{
    在\textit{EBSN}中,事件描述(\textit{event description})提供了相当重要的信息,例如在事件推荐和安排中起到
    衡量事件间的相似度。而毫无疑问,事件描述对事件参与人数也有很大影响。特别是在其他因素被限制的情况下(时间,地点,人数等),
    事件描述给了事件举办者最大的自由度来使他的事件在其他同类事件中脱颖而出。而什么样的事件描述算是一份好的事件描述,其
    衡量标准也是相当复杂的。因为事件描述的好坏评价是件相当主观的事情,所谓吾之蜜糖,彼之毒药。这也为如何利用好事件
    描述里包含的信息带来了困难。而直接去定量的研究事件描述对事件参与人数,或成功的影响的论文更是少之又少,而这个问题,
    正是本文想去解答的。
}
\par{
    总的来说,本文尝试解答如下三个问题:1)事件描述对事件参与人数的影响。2)如何仅通过事件描述来预测事件参与人数。3)如何
    改进事件描述,使之能吸引更多的人参加。 其中,第一问题的答案是符合直觉的:事件描述对事件参与人数有影响,而且影响程度与
    不同的词语有关,例如包含\textit{party}的描述比包含\textit{church}的描述的参与人数要高不少。在研究第二个问题上,
    ,由于对象是自然语言,我使用了一个经典的带卷积层的前向神经网络和词向量来做预测。在解答第三个问题上,我使用了一个改进的
    差分自动编码器(\textit{VAE})的序列到序列(\textit{seq2seq})网络来生成对抗样本,使用第二个问题中的前向神经网络来
    做预测。
}

\end{document}
