\documentclass[]{template}

\begin{document}
\section{绪论}
绪论:绪论相当于论文的开头,它是三段式论文的第一段(后二段是本论和结论)。 绪论与摘要写法不完全相同,摘要要写得高度概括、简略,绪论可以稍加具体一些, 文字以 1000 字左右为宜。绪论一般应包括以下几个内容:
\textcircled{1}为什么要写这篇论文,要解决什么问题,主要观点是什么。

\textcircled{2}对本论文研究主题范围内已有文献的评述(包括与课题相关的历史的回顾,资 料来源、性质及运用情况等)。 

\textcircled{3}说明本论文所要解决的问题,所采用的研究手段、方式、方法。明确研究工作 的界限和规模。 

\textcircled{4}概括论文的主要工作内容。  
 
\subsection{标题 2}
\subsubsection{标题 3}
图、表、公式等一律用阿拉伯数字分章连续编号,如 图 1-3、表 2-1、( 3-2)等。 图、表、公式等与正文之间间隔 0.5 行。 
 
图应有图题,表应有表题,并分别置于图号和表号之后,图号和图题应置于图下 方的居中位置,表号和表题应置于表上方的居中位置。引用图或表应在图题或表题右 上角标出文献来源。 

    若图或表中有附注,采用英文小写字母顺序编号,附注写在图或表的下方。 \footnote{脚注是对文中有关内容的解释、说明或补充,使用上角标(序号\textcircled{1}、\textcircled{2}…)标注,脚注可用小号字(一般小五号 宋体)列在相应正文同一页最下部并与正文部分用细线(版面宽度的 1/4 长)隔开。(删除脚注的方法:直接删除 正文中的脚注编号即可) }
    图应有图题,表应有表题,并分别置于图号和表号之后,图号和图题应置于图下 方的居中位置,表号和表题应置于表上方的居中位置。引用图或表应在图题或表题右 上角标出文献来源。 图应有图题,表应有表题,并分别置于图号和表号之后,图号和图题应置于图下 方的居中位置,表号和表题应置于表上方的居中位置。引用图或表应在图题或表题右 上角标出文献来源。 图应有图题,表应有表题,并分别置于图号和表号之后,图号和图题应置于图下 方的居中位置,表号和表题应置于表上方的居中位置。引用图或表应在图题或表题右 上角标出文献来源。 图应有图题,表应有表题,并分别置于图号和表号之后,图号和图题应置于图下 方的居中位置,表号和表题应置于表上方的居中位置。引用图或表应在图题或表题右 上角标出文献来源。 图应有图题,表应有表题,并分别置于图号和表号之后,图号和图题应置于图下 方的居中位置,表号和表题应置于表上方的居中位置。引用图或表应在图题或表题右 上角标出文献来源。 图应有图题,表应有表题,并分别置于图号和表号之后,图号和图题应置于图下 方的居中位置,表号和表题应置于表上方的居中位置。引用图或表应在图题或表题右 上角标出文献来源。 图应有图题,表应有表题,并分别置于图号和表号之后,图号和图题应置于图下 方的居中位置,表号和表题应置于表上方的居中位置。引用图或表应在图题或表题右 上角标出文献来源。 图应有图题,表应有表题,并分别置于图号和表号之后,图号和图题应置于图下 方的居中位置,表号和表题应置于表上方的居中位置。引用图或表应在图题或表题右 上角标出文献来源。 图应有图题,表应有表题,并分别置于图号和表号之后,图号和图题应置于图下 方的居中位置,表号和表题应置于表上方的居中位置。引用图或表应在图题或表题右 上角标出文献来源。 图应有图题,表应有表题,并分别置于图号和表号之后,图号和图题应置于图下 方的居中位置,表号和表题应置于表上方的居中位置。引用图或表应在图题或表题右 上角标出文献来源。 图应有图题,表应有表题,并分别置于图号和表号之后,图号和图题应置于图下 方的居中位置,表号和表题应置于表上方的居中位置。引用图或表应在图题或表题右 上角标出文献来源。 图应有图题,表应有表题,并分别置于图号和表号之后,图号和图题应置于图下 方的居中位置,表号和表题应置于表上方的居中位置。引用图或表应在图题或表题右 上角标出文献来源。 
\newpage

图、表、公式等一律用阿拉伯数字分章连续编号,如 图 1-3、表 2-1、( 3-2)等。 图、表、公式等与正文之间间隔 0.5 行。 

图应有图题,表应有表题,并分别置于图号和表号之后,图号和图题应置于图下 方的居中位置,表号和表题应置于表上方的居中位置。引用图或表应在图题或表题右 上角标出文献来源。 

\end{document} 

 